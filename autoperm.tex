\documentclass{article}

\title{Autoperm Cipher}
\date{8th December 2019}
\author{Alastair Horn}

\RequirePackage[
 margin=1in,
 headheight=13.6pt]{geometry}

 \usepackage{amsmath}

\setlength{\parindent}{0cm}
\setlength{\parskip}{1em}

\begin{document}

\maketitle

\section*{Introduction}

While I was going to sleep last night this idea for 
a cipher popped into my head. As any aspiring nerd
would, I hopped half-asleep out of bed and
wrote the idea down.

What follows is a short explanation of the cipher.
Accopanying code is provided for enciphering and deciphering.

\section*{The Cipher}

We are given as plaintext a string of letters A-Z
which will here be represented by numbers 1-26 in
alphabetical order. Let the plaintext be $a_1, a_2, \ldots, a_n$,
and the target ciphertext be $b_1, b_2, \ldots, b_n$.

The keys are two permutations $\sigma_0$, $\tau_0$ of
$\{1, 2, \ldots, 26\}$. The keys and the plaintext together produce
sequences of permutations as follows:
\begin{align*}
\sigma_{n+1} &= \sigma_n \circ (a_{2n} \ \ a_{2n+1}) \\
\tau_{n+1} &= \tau_n \circ (a_{2n} \ \ a_{2n+1})
\end{align*}
for all $n$.

\textit{Round brackets are used here to write cycles,
and $\circ$ is used to compose functions.}

And of course we have a way to generate ciphertext from the
plaintext and the sequences of permutations:
\begin{align*}
b_{2n} &= \sigma_n(a_{2n}) \\
b_{2n+1} &= \tau_n (a_{2n+1})
\end{align*}

\end{document}
