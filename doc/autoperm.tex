% This is a LaTeX document, to be compiled something like pdflatex (for which
% you will need a TeX distribution such as TeXLive)

\documentclass{article}
\title{Autoperm Cipher}
\author{Alastair Horn}

\usepackage[
 margin=1in,
 headheight=13.6pt]{geometry}

\usepackage{amsmath}

\usepackage{parskip}

\begin{document}

\maketitle

\section*{Introduction}

This idea for a cipher popped into my head recently.

What follows is a short explanation of the cipher. Accompanying code is provided
for enciphering and deciphering.

\section*{The Cipher}

We are given as plaintext a string of letters A-Z which will here be represented
by numbers 1-26 in alphabetical order. Let the plaintext be \(m\) letters in
length, denoted by \(a_1, a_2, \dotsc, a_m\), and the target ciphertext be
\(b_1, b_2, \dotsc, b_m\).

The keys are two permutations \(\sigma_0\), \(\tau_0\) of
\(\{1, 2, \dotsc, 26\}\). The keys and the plaintext together produce sequences
of permutations as follows:
\begin{alignat*}2
 \sigma_{n + 1} &= \sigma_n && \circ (a_{2n}\ a_{2n + 1}) \\
 \tau_{n + 1}   &= \tau_n   && \circ (a_{2n}\ a_{2n + 1})
\end{alignat*}
for all \(n\).

\emph{Round brackets are used here to write cycles, and \(\circ\) is used to
compose functions.}

And of course we have a way to generate ciphertext from the plaintext and the
sequences of permutations:
\begin{align*}
  b_{2n}     &= \sigma_n(a_{2n}) \\
  b_{2n + 1} &= \tau_n (a_{2n + 1})
\end{align*}

\section*{Cryptanalysis and Known Weaknesses}

Over short distances within the plaintext, letters appearing more than once are 
likely to produce a similar result in the ciphertext.
Consider for example some plaintext letters
$a_{2n}, a_{2n+1}, a_{2n+2}, a_{2n+3}$, with $a_{2n+1} = a_{2n+2}$. We have
\begin{align*}
  b_{2n}   &= \sigma_n(a_{2n}) \\
  b_{2n+1} &= \tau_n(a_{2n+1}) \\
  b_{2n+2} &= \sigma_{n+1}(a_{2n+2})  \\
           &= (\sigma_n \circ (a_{2n}\ a_{2n+1}))(a_{2n+1}) \\
           &= b_{2n}
\end{align*}
and other similar formations of letters cropping up again can produce similar
arrangements.

\end{document}
